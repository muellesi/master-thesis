In diesem Kapitel wird nachfolgend das grundlegende Konzept der Gestenerkennung von den Rohdaten bis zur Klassifizierung der eigentlichen Handgeste besprochen.

\section{Requirements}
In this thesis we develop a system for 
\subsection{Software}
In principle, it is possible to classify hand gestures directly from recorded depth information using machine learning. As in \cite{Molchanov2016}, a combination of a CNN and a downstream RNN can be used for this purpose. In this variant \todo{better formulation}, however, the structure of the network and the number and type of classes are strictly coupled. Adding, removing or changing a class would force us to retrain large parts of the network.
A further problem of this direct approach is that a sufficient amount of training data, consisting of image sequences and corresponding class labels, must be available for a reliable classification.

If new hand gestures with few example data are to be learned, and/or in the ideal case after unique Vormachen, it makes sense to divide the data processing into two sequential steps:

First, we generate a per-frame estimate of the current hand pose from the 2.5-dimensional input data (RGB-D). 


A possible time dependence of the respective hand gesture is not yet taken into account. 




\begin{figure}
	\centering
	\input{Ressourcen/software-net-structure-blackbox.tex}
	\caption{}
	\label{fig:software-net-structure-blackbox}
\end{figure}


\subsection{Hardware}
- Realsense
- Testrack