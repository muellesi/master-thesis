In diesem Kapitel wird nachfolgend das grundlegende Konzept der Gestenerkennung von den Rohdaten bis zur Klassifizierung der eigentlichen Handgeste besprochen.

\section{Systementwurf}

\subsection{Software}
Grundsätzlich ist es möglich, mit Hilfe von maschinellem Lernen eine Klassifizierung von Handgesten direkt aus aufgezeichneten Tiefeninformationen durchzuführen. Hierfür kann wie in \cite{Molchanov2016} eine Kombination aus einem CNN und einem nachgeschalteten RNN genutzt werden. In dieser Variante \todo{bessere Formulierung} sind allerdings die Struktur des Netzes sowie Anzahl und Art der Klassen streng gekoppelt. Hinzufügen oder Entfernen, beziehungsweise Ändern einer Klasse erfordert somit in jedem Fall ein erneutes Training des Netzes.
Ein weiteres Problem der genannten Vorgehensweise ist, dass für eine zuverlässige Klassifizierung eine ausreichende Menge an Trainingsdaten, bestehend aus Bildsequenzen und zugehörigen Klassenlabels zur Verfügung stehen muss. Soll das System allerdings im laufenden Betrieb mit neuen Klassen erweitert werden können, ist die Beschaffung dieser Daten nicht ohne weiteres möglich. Der Benutzer müsste in diesem Fall eine Geste zu häufig wiederholen.

Sollen neue Handgesten mit wenigen Beispieldaten, beziehungsweise im Idealfall nach einmaligem Vormachen erlernt werden, ist es sinnvoll, die Verarbeitung der Daten in zwei aufeinanderfolgenden Schritten vorzunehmen:

Zunächst wird aus den 2,5-dimensionalen Eingangsdaten (RGB-D) eine Handpose geschätzt. Diese kann je nach geforderter Genauigkeit in einem Vektor mit 21 oder mehr Elementen dargestellt werden und direkt (). Eine eventuelle Zeitabhängigkeit der jeweiligen Handgeste findet dabei noch keine Beachtung. 
Im zweiten Schritt wird anschließend

Eine akkumulierte Anzahl aufeinanderfolgender Handposen kann anschließend in einem weiteren Netz, dem Gestenklassifikator einer zuvor aufgezeichneten Geste zugeordnet werden.


\begin{figure}
	\centering
	\input{Ressourcen/software-net-structure-blackbox.tex}
	\caption{}
	\label{fig:software-net-structure-blackbox}
\end{figure}


\subsection{Hardware}
