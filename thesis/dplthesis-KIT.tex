%##################################################################################

\documentclass[12pt,a4paper,twoside]{report}
\usepackage[T1]{fontenc}
%\documentclass[12pt,a4paper,oneside]{report}
\usepackage{graphicx}
\usepackage[ngerman, english]{babel} %-- uncomment this to get english titles
%\usepackage[ngerman]{babel}
\usepackage{german,a4}
%\usepackage{picins}
%\usepackage{epsfig}
\usepackage{fancyhdr}			% for nice header and footer
\usepackage{hyperref}			% references in pdf
\usepackage[sf]{titlesec}	%customization of \chapter Titles for appendices
\usepackage{textcomp}
\usepackage{makeidx}
\usepackage{ngerman}			% german Umlauts
\usepackage[utf8]{inputenc}
\usepackage{booktabs}                               % necessary for tabulars
\usepackage[squaren]{SIunits}
\usepackage{tikz}
\usepackage[numbers,square]{natbib}
\usepackage{pgfplots}
\usetikzlibrary{shapes,arrows}
%\usepackage{wrapfig} 
%\usepackage[pdftex]{graphicx}
%\usepackage{ifthen}  
%\usepackage{booktabs} % fancy tables
%\usepackage[titletoc]{appendix} % custom naming of appendices
\usepackage{amssymb, amsmath, amsthm} % for equations & eqref
%\usepackage{listings} \lstset{basicstyle=\tiny\ttfamily, numbers=left, escapeinside={(*}{*)}, captionpos=b}
%\usepackage{dirtree}
\usepackage{acronym}
\usepackage{caption}
\usepackage{subcaption}
\usepackage{epigraph}
\usepackage{csquotes}
\usepackage{braket}
\usepackage{etoolbox}
\AtBeginEnvironment{quote}{\singlespacing\small}

%get bigger \par with one empty line
\newcommand{\mypar}{\par\medskip}

%TODO line
\newcommand{\writeTodo}{1}
\newcommand{\todo}[1]{
	\ifdefined \writeTodo
		\mypar\textbf{\textcolor{KITgreen}{TODO: }\textcolor{red}{#1}}\mypar
	\fi
} 


\newcommand{\matr}[1]{\mathbf{#1}}
%\newcommand{\matr}[1]{#1}
%\newcommand{\matr}[1]{\bm{#1}}     % ISO complying version

%\theoremstyle{plain}% default
\theoremstyle{definition}
\newtheorem{definition}{Definition}

% Abbkürzungsverzeichnis einfügen
%%%%%%%%%%%%%%%%%%%%%%%%%%%%%%%%%%%%%%%%%%%%%%%%%%%%%%%%% 
\usepackage{nomencl}
\let\abbrev\nomenclature
\renewcommand{\nomname}{Abkürzungsverzeichnis}
\setlength{\nomlabelwidth}{.25\hsize}
\renewcommand{\nomlabel}[1]{\hypertarget{nomencl-#1}{#1} }%\dotfill}
\setlength{\nomitemsep}{-\parsep}
\makenomenclature
\newcommand{\markup}[1]{\underline{\hyperlink{nomencl-#1}{#1}}}

% Farben
\usepackage{color}
\definecolor{KITgreen}{rgb}{0, .61, .51} 
\definecolor{KITbluegrey}{rgb}{.27, .39, .67} 
\definecolor{KITgrey}{rgb}{.49, .49, .49} 

% =====================================================
% Dokumenten-Platzhalter
% =====================================================
% =====================================================
% Dokumenten Platzhalter
% =====================================================


\newcommand{\titelderarbeit}{Innovative selbstlernende Gestensteuerung im Automotivbereich}
\newcommand{\artderarbeit}{Masterarbeit}

\newcommand{\diplomandprefix}{cand. el.}
\newcommand{\diplomand}{Simon Müller}

\newcommand{\betreuerA}{Marco Stang}
\newcommand{\betreuerB}{Simon Stock}

 
%\newcommand{\nameprefix}{}{}    %use no nameprefix 
\newcommand{\nameprefix}{M. Sc. }
\newcommand{\docauthor}{Simon Müller}

%\newcommand{\nameprefixb}{}{}    %use no nameprefix 
\newcommand{\nameprefixb}{M. Sc.}  
\newcommand{\docauthorb}{}

\newcommand{\nameprefixc}{}{}    %use no nameprefix 
%\newcommand{\nameprefixc}{Dipl.-Ing.}
\newcommand{\docauthorc}{}

%\newcommand{\betreuerA}{nn}
%\newcommand{\betreuerB}{nn}
\newcommand{\abgabe}{8 April 2020}
\newcommand{\versionierungsnr}{v1.0}
\newcommand{\fussnoteninhalt}{}
\newcommand{\leerzeichen}{ }

%% english version
%\newcommand{\Dachorganisation}{Karlsruhe Institute of Technology - KIT}
%\newcommand{\Institut}{Institute for Information Processing 
%												Technology - ITIV}
												
% deutsche version
\newcommand{\Dachorganisation}{Karlsruher Institut für Technologie - KIT}
\newcommand{\Institut}{Institut für Technik der Informationsverarbeitung - ITIV}

% ====================================================
% Formatierung des Dokuments
% ====================================================
\input{00_special_pages/formats}

% =====================================================
% Inhalt der Titelseite definieren
% =====================================================

\makeindex
\newcommand{\Idx}[1]{#1 \index{#1}}

\hyphenation{
EVITA
}

% =====================================================
% Zeichen für Copyright, Trademark, Registerd, ...
% =====================================================
\def\TReg{\textsuperscript{\textregistered}}
\def\TCop{\textsuperscript{\textcopyright}}
\def\TTra{\textsuperscript{\texttrademark}}

% =====================================================
% selbs definierte Zeichen
% =====================================================
\def\zB{z.\,B. }
\def\uvm{u.\,v.\,m.}


\begin{document}

% =====================================================
% Put Titel in English
% ===================================================== 
\input{00_special_pages/titelseite_en}
    \parindent=0pt
    %\sloppypar
    \linespread{1.2}
    \thispagestyle{plain}
    %\frontmatter
    %\maketitle

% =====================================================
% Put Titel in Deutsch
% ===================================================== 
%\input{00_special_pages/titelseite}
%    \parindent=0pt
%    %\sloppypar
%    \linespread{1.2}
%    \thispagestyle{plain}
%    %\frontmatter
%    %\maketitle
%    \cleardoublepage
    
% =====================================================
% Abstract
% =====================================================     
	\begin{abstract}
		Hier kommt die Zusammenfassung ...


\cleardoublepage

\chapter*{Urheberrecht}

ARM\TReg, AMBA\TReg, AXI\TTra, Cortex\TTra, TrustZone\TTra, SecurCore\TTra  , DSTREAM\TTra und weitere im Text erwähnte ARM-Produkte sowie die entsprechenden Logos sind Marken oder eingetragene Marken der Advanced RISC Machines Ltd.\par
\vspace{0.5cm}
Xilinx\TReg, Zynq\TTra und weitere im Text erwähnte Xilinx-Produkte sowie die entsprechenden Logos sind Marken oder eingetragene Marken der Xilinx Inc.\par
\vspace{0.5cm}
	\end{abstract}
	\cleardoublepage
% =====================================================
% Signaturepage
% ===================================================== 
    \input{00_special_pages/erklaerung}
	\cleardoublepage
	
	Thanks at 
\cite{Tompson2014, Yuan2017, Sun2015, Yuan2017b, GarciaHernando2017, Yuan2017c} for providing their datasets to the public.
\todo{Schöner machen!}
% =====================================================
% TOC
% ===================================================== 
    \tableofcontents
    %\clearpage
    \cleardoublepage

% =====================================================
% Main Chapters
% ===================================================== 
    %\mainmatter
    \pagenumbering{arabic}
    \setcounter{page}{1}
    \pagestyle{fancy}
 \normalsize

    
    \chapter{Introduction}

    \label{ch:Einleitung}    
    	\todo{bislang nur automatisch übersetzt}
With the increasing popularity of virtual assistants, alternative operating methods for user interfaces are increasingly being used in everyday life in the automotive sector as well. Many assistants of the current generation, however, only provide voice input.

However, especially with regard to accessibility, it is desirable to provide other input options in addition to pure voice input. However, even for people without limitations of acoustic perception, speech input is not always the optimal input method. An example of this is the multimedia control in an automobile - here voice input may already be restricted by the music being played to such an extent that traditional input methods such as control panels or touchscreens must be used. 

However, since inputs at the touch of a button or touch display can distract drivers from road traffic, additional input options that do not require the driver to turn away his eyes could additionally increase safety.

One possible solution to these problems is gesture control, in which commands are assigned to specific hand gestures. Hand gestures can be static, but can also include complex motion sequences, which are usually pre-programmed. Due to very different usage patterns, a system in which gestures can be freely defined by the user would be advantageous. 

Within the scope of this work, an innovative self-learning gesture control system for automobiles is to be developed, which makes it possible to program arbitrary hand gestures by demonstration. In a first step, the system can be pre-trained with the help of several recorded videos of different hand gestures, but the goal should be a one-shot learning, so that the user can define any gestures by showing them once.

For this purpose, a two-part system consisting of a Convolutional Neural Network (CNN) and a classifier will be used. The purpose of the CNN will be to map a depth image of the hand gesture recorded by a 3D camera (Intel Realsense) to the virtual representation of a hand, thus decisively reducing the dimensionality of the input data. The pose (or sequence of poses) is then assigned to a user-defined function using a classifier. The later goal of one-shot learning must be taken into account, which excludes algorithms that require a lot of training data. Since moving gestures may also be possible, a classifier is also required that can take this into account and assign the chronological sequence of several poses to a single gesture.

\section{Current state of research}
Research in the field of computer vision, in particular object recognition and classification, has been a strong focus in recent years \cite{FeiFei}. Especially in the field of gesture recognition there are numerous studies that investigate different strategies for segmentation and interpretation of hand and whole body gestures \cite{Zimmermann2017,Sato2001,Supancic2018,Tompson2014,Zhang2016,OhnBar2014,Ge2019,Keskin2012,Li2013,Jones2002}. 


For the segmentation two criteria usually represent a special challenge in the cited work:

Firstly, for a successful extraction of the hands from an RGB image, relevant image areas must be separated from non-relevant areas. A simple approach about the skin color as in \cite{Sato2001} is not always sufficient even when viewing in different color spaces through different skin types and light situations, which is why \cite{Zhang2016} and \cite{Li2013} additionally use a combination of different techniques such as background subtraction, viewing the texture and grouping in superpixels. Often, however, as in \cite{Sridhar2013}, an optimized background with few or no skin tones is used.

On the other hand, self occlusion during the subsequent gesture recognition is a problem that can make the recognition of certain poses more difficult. Here a fusion of depth information and RGB camera images can bring decisive advantages \cite{Keskin2012}.

The subsequent classification of poses into gestures is not dealt with further in most works, since often only pose recognition is the goal.
    \cleardoublepage
    
  	
    
    \chapter{Theoretical Basis}
    \label{ch:Grundlagen}
		\section { Computer Vision }
	\subsection { Kameraparameter }
	\subsection { Segmentierung }
		

	
\section { Machine Learning }
	In den letzen Jahren gab es stetige Fortschritte in der Leistungsfähigkeit von Rechenhardware und damit einhergehend im Bereich des maschinellen Lernens (en. machine learning).
	In diesem Abschnitt werden kurz die in der Arbeit genutzten machine-learning Techniken aufgezeigt und erläutert.
	
	\subsection { Neuronale Netze }
	Künstliche Neuronale Netze (en.: artificial neural networks, kurz: ANN) sind der biologischen Funktionsweise menschlicher Nervenbahnen nachempfunden. 
	
		\subsubsection { Das Perzeptron }
		Das einlagige Perzeptron, der einfachste Fall eines neuronalen Netzes, besteht aus drei Schichten von jeweils beliebig vielen Recheneinheiten, die als Neuronen bezeichnet werden. Alle Neuronen sind dabei mit allen Neuronen der nachfolgenden Schicht über eine gewichtete Verbindung verknüpft. 
		
		Die erste Schicht übernimmt im einlagigen Perzeptron die Rolle des Eingangs. Die zweite Lage wird als "`hidden layer"' bezeichnet und die dritte Schicht stellt die berechneten Ausgangsdaten zur Verfügung.
		
		Jede Recheneinheit nimmt die Daten $o_{j-1} = o_i$ aus den vorhergehenden Neuronen entgegen und berechnet einen vorläufigen Ausgangswert $p_j$ nach Gleichung \ref{eq:perceptron_simple}. 
		
		\begin{equation}
			\label{eq:perceptron_simple}
			\text{net}_j = \sum_{i=1}^{n} w_{ij} \cdot o_i + b
		\end{equation}
		
		Die Gewichte $w_{xy}$ und Bias-Werte $b_x$ werden im Trainingsprozess anhand der Trainingsdaten optimiert und ändern sich nach dem Training nicht mehr.
		
		
		Die Summe der Produkte wird anschließend über eine nichtlineare Aktivierungsfunktion gefiltert und daraufhin in die nächste Schicht weitergereicht:
		
		\begin{equation}
		\label{eq:perceptron_act}
		o_j = \varphi\left(\text{net}_j\right)
		\end{equation}
		
		
		 
		\subsubsection { Aktivierungsfunktionen }
		Da die grundlegenden Rechenoperationen in einem neuronalen Netz linearer Natur sind, muss eine zusätzliche Nichtlinearität eingeführt werden, um auch nichtlineare Zusammenhänge erlernen zu können. 
		Hierfür werden unterschiedliche Aktivierungsfunktionen $\varphi$ eingesetzt, von denen einige häufig genutzte Funktionen im Folgenden erläutert werden.\\
		
		
		\textbf{Schwellenwertfunktion}
		Die Schwellenwertfunktion ist die ursprüngliche Aktivierungsfunktion für das Perzeptron nach \cite{McCulloch1943} und besitzt lediglich $0$ und $1$ als mögliche Ausgangswerte. Sie ist mit dem Schwellenwert $\epsilon$ definiert zu 
		\begin{equation}
		\label{eq:acti_sw}
		o_j = \left\{
		\begin{array}{ll}
		1\text{, wenn } \text{net}_j > \epsilon \\
		0 \text{ sonst}\\
		\end{array}
		\right.
		\end{equation}
		\textbf{Sigmoid}
			Die Sigmoid-Funktion ist mit einem variablen Steigungsparameter $a$ definiert zu 
			\begin{equation}
			\varphi\left(\text{net}_j\right) = \frac{1}{1+\exp(-a \cdot \text{net}_j)}
			\end{equation}
			Sie wird häufig anstatt der Schwellenwertfunktion genutzt, da sie stetig differenzierbar und somit gut geeignet für häufig genutzte Trainingsverfahren wie Gradient Descend ist.\\
			\begin{figure}[ht]
				\centering
				\begin{tikzpicture}
				\begin{axis}[
				domain=-200:200,
				xmin=-10, xmax=10,
				ymin=-1.5, ymax=1.5,
				samples=401,
				axis y line=center,
				axis x line=middle,
				]
				\addplot+[mark=none] {1/(1 + exp(-x)};
				\end{axis}
				\end{tikzpicture}
				\caption{Die Sigmoid-Funktion begrenzt die Ausgangswerte wie auch die Schwellenwertfunktion auf den Bereich [0, 1].}
				\label{fig:sigmoid_plot}
			\end{figure}


		\textbf{ReLu}
		Die Rectifying linear unit (kurz: ReLu) ist eine weitere Form der Aktivierungsfunktion, die insbesondere in Deep Neural Networks und Convolutional Neural Networks eingesetzt wird. Sie ist definiert zu
		\begin{equation}
			\label{eq:relu_def}
			\varphi(\text{net}_j) = \max(\text{net}_j, 0)
		\end{equation}
		womit negative Werte abgeschnitten werden. Im Vergleich zur Schwellenwert- bzw. Sigmoidfunktion führen große Eingangswerte hier nicht zur Sättigung (und damit kleinem Gradienten), was insbesondere in Gradientenverfahren wie in Abschnitt \ref{sec:gradient-descend} von Vorteil ist. \\
		
		\begin{figure}[ht]
			\centering
		\begin{tikzpicture}
		\begin{axis}[
		domain=-200:200,
		xmin=-10, xmax=10,
		ymin=-10, ymax=10,
		samples=401,
		axis y line=center,
		axis x line=middle,
		]
		\addplot+[mark=none] {max(x, 0)};
		\end{axis}
		\end{tikzpicture}
		\caption{Durch die ReLu-Aktivierungsfunktion werden negative Werte abgeschnitten.}
		\label{fig:relu_plot}
		\end{figure}

	
	\subsection { Spezialfälle Neuronaler Netze}
		\subsubsection { Convolutional Neural Networks (CNN) }
		Faltungsnetzwerke (en.: Convolutional neural networks, kurz CNN) sind ein Spezialfall der neuronalen Netze, der insbesondere für die Verarbeitung von höherdimensionalen Strukturen wie Bilder oder zeitliche Abfolgen von Daten geeignet ist. 
		
		In einem CNN werden zusätzlich zu den oben beschriebenen "`Dense-"' oder "`Fully-Connected-"' weitere Schichten eingesetzt, die Faltungsoperationen auf den Daten durchführen.
		
		Anstatt einer einfachen Multiplikation erfolgt in jeder Faltungsschicht eine Faltung des Eingangstensors mit einer Faltungsmatrix.
		
		Im Fall eines 2D-CNN handelt es sich bei den Eingangsdaten um eine 2D-Matrix. Eine Faltungsschicht enthält mehrere ebenfalls zweidimensionale Matrizen, die über die Eingangsmatrix geschoben werden. 
		
		
		\todo{Bild für Faltung}
		
		
		\subsubsection { Recurrent Neural Networks (RNN) }
		\subsubsection { Long Short Term Memory (LSTM) }
	
	\subsection { Trainingsmethoden }
		
		\subsubsection{Gradient Descend}
		\label{sec:gradient-descend}
		
		\subsubsection{ADAM}
		\todo{Link zum Paper ist in tensorflow source von adam optimizer zu finden}
		
	\subsection{ One Shot Learning }
	Ein Problem der bisher gezeigten Machine Learning-Methoden ist, dass sehr viele Traningsdaten benötigt werden, um die Netze ausreichend zu trainieren. In Anwendungsfällen, in denen die Trainingsdaten erst in der Benutzerinteraktion zur Verfügung stehen können jedoch häufig nicht ausreichend viele Daten gesammelt werden. Eine Möglichkeit zur Umgehung des Problems ist es, keine direkte Klassifizierung durchzuführen sondern ein Netz darauf zu trainieren, die Eingangsdaten mit einem zuvor gespeicherten Datensatz zu vergleichen. \todo{weiter}
	
\section { Anatomie der menschlichen Hand }
	Die Bewegungsfreiheit der einzelnen Handglieder unterliegt anatomischen Beschränkungen, die in der Posenschätzung nützlich sein können, um die Güte der Schätzung zu beurteilen und mit einem passenden Modell entsprechend verfeinern zu können \cite{Melax5222017}.
	
	\begin{figure}
		\centering
		\includegraphics[width=0.7\linewidth]{Ressourcen/malik2018_hand_model}
		\caption[Handmodell nach \cite{Malik2018b}]{In \cite{Malik2018b} wird ein Modell ähnlich dem oben stehenden (Quelle: \cite{Malik2018b}) genutzt, in dem die vollständige Handpose durch 21, bzw. 22 (Gelenk-) Koordinaten bestimmt ist.}
		\label{fig:malik2018handmodel}
	\end{figure}
	
	
	\subsection{title}
	
    \cleardoublepage

    \chapter{Related Work}
			\section{Machine Learning}
			\section{Computer Vision}
				\subsection{Pose estimation}
				\subsection{Gesture Classification}
				
	

	\chapter{Conception and Requirements}
	\label{ch:Konzept}  	
		In diesem Kapitel wird nachfolgend das grundlegende Konzept der Gestenerkennung von den Rohdaten bis zur Klassifizierung der eigentlichen Handgeste besprochen.

\section{Requirements}
In this thesis we develop a system for 
\subsection{Software}
In principle, it is possible to classify hand gestures directly from recorded depth information using machine learning. As in \cite{Molchanov2016}, a combination of a CNN and a downstream RNN can be used for this purpose. In this variant \todo{better formulation}, however, the structure of the network and the number and type of classes are strictly coupled. Adding, removing or changing a class would force us to retrain large parts of the network.
A further problem of this direct approach is that a sufficient amount of training data, consisting of image sequences and corresponding class labels, must be available for a reliable classification.

If new hand gestures with few example data are to be learned, and/or in the ideal case after unique Vormachen, it makes sense to divide the data processing into two sequential steps:

First, we generate a per-frame estimate of the current hand pose from the 2.5-dimensional input data (RGB-D). 


A possible time dependence of the respective hand gesture is not yet taken into account. 




\begin{figure}
	\centering
	\input{Ressourcen/software-net-structure-blackbox.tex}
	\caption{}
	\label{fig:software-net-structure-blackbox}
\end{figure}


\subsection{Hardware}
- Realsense
- Testrack
    \cleardoublepage    
    
    \chapter{Implementation}
    \label{ch:implementierung}   
   		\section{Hand pose estimation}
For the task of estimating the user's hand pose, two different hand pose regressors are implemented and evaluated.

For both regressors, ImageNetV2 \cite{Sandler} is implemented as an encoder to transfer the given depth data into a latent feature space. 

\subsection{Structural training}
As described in the previous section, the hand pose module uses an encoder to convert the high-dimensional input data into a latent feature space. 
Using the principle of transfer learning, the encoder can be trained separately in order to improve the overall performance of the pose regression network. This pretraining can involve any task which requires the encoder to transfer depth images of hands into some latent feature space.

Since depth images of human hands are prone to severe self-occlusion, they are hard to label precisely which limits the amount of available data. For the pretraining it is therefore preferable to train for a task which does not depend on potentially noisy labels. 

Training the encoder as part of an autoencoder is ideal under this circumstance since the target data is a direct copy (with some automatic modifications) of the input data. This simplifies the creation of additional training data.

On that basis an autoencoder is implemented, using MobileNetv2 \cite{Sandler} as its encoder part. With an input resolution of $224 \times 224$ pixels, the encoder transforms incoming depth maps into the latent feature space, consisting of $1280$ feature maps with a size of $7\times7$ each. 



The Autoencoder is trained on a custom dataset, consisting of short sequences of hand movements which are split into single frames. Its target is a segmented version of each depth map, containing only the hand and parts of the arm. 

In order to provide a wider range of backgrounds, a second dataset is then captured which contains no hands at all. By segmenting the hands from the first dataset and combining them with a random background from the second dataset, the variation of the dataset can be increased. Since the dataset consists exclusively of depth maps and the hand is guaranteed to be the closest object in each map, segmentation can be done by simply clipping values that are larger than a known threshold. The segmented hand images can then be used for both, the mentioned augmentation task and as the target for the autoencoder. 

\begin{figure}
	\centering
	\includegraphics[width=0.7\linewidth]{Ressourcen/ae_example}
	\caption[Structural training]{Structural training. \textbf{Left}: Segmented depth map, containing only the hand and parts of the arm. \textbf{Center}: Segmented image with random background. \textbf{Right}: Reconstruction by the Autoenoder during training. }
	\label{fig:aeexample}
\end{figure}


 Additionally, classic augmentations methods like stretching, shearing and zooming/cropping are used to further augment the available data. As a final step, random noise is added to the input image, preserving 

\subsection{Training and Hyperparameter Tuning}


\section{Gesture Classification}
\section{Physical Construction}

   		
    \chapter{Results and Evaluation}
	\label{ch:Bewertung}  
		\input{05_diskussion/diskussion.tex}
    \cleardoublepage    

    \chapter{Conclusion and Outlook}

    \chapter{Appendix}
    \cleardoublepage
    
% =====================================================
% Bibliography
% ===================================================== 
%\nocite{*}

%====== DATASETS
\nocite{Tompson2014, Yuan2017, Sun2015, Yuan2017b, GarciaHernando2017, Yuan2017c}

%\addcontentsline{toc}{chapter}{Literaturverzeichnis, in ToC löschen}
%\addcontentsline{toc}{chapter}{Literature}
%\bibliographystyle{dinat}        % use a DIN style for the bibliography
%\bibliographystyle{plain}        % use a DIN style for the bibliography
%\bibliographystyle{abbrv}			%mit nummern in []
%\bibliographystyle{plain}			%mit nummern in []
%\bibliographystyle{alpha}			%3 Buchstaben + Jahr
%\bibliographystyle{alphadin}
\bibliographystyle{IEEEtran}

    % use external Bib-File, 
		\bibliography{99_bib/standardbib}
		\cleardoublepage
% =====================================================
% Appendix
% ===================================================== 
    \appendix

%    \input{95_anhang/anhang}
		\cleardoublepage

    \listoftables
    	\cleardoublepage
    \listoffigures
    	\cleardoublepage
    \printindex
    	\cleardoublepage
    \printnomenclature
    \renewcommand{\leftmark}{\uppercase{Abkürzungsverzeichnis}}
    	\cleardoublepage

    \section{Acknowledgement}
    Writing my master thesis at ITIV was a new and previously unknown experience. 
    I would like to thank my supervisors, Marco Stang and Simon Stock for their ever helpful advice and guidance and the many discussions we had about possible solutions.
    I would also like to thank the Machine Learning Community for their constant efforts to improve the tools and datasets used in this work. Without the many people reporting and fixing bugs, developing new functionalities, labeling datasets and most importantly: providing their work to everyone else, this thesis would not have been possible to do in the given timespan.
    
    Special thanks goes to the publishers in \cite{Tompson2014, Yuan2017, Sun2015, Yuan2017b, GarciaHernando2017, Yuan2017c} for providing their datasets for academic research.

\todo{Schöner machen!}

\end{document}

